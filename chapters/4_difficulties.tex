\setchapterpreamble[u]{\margintoc}
\chapter{Why writing documentation is boring}
\label{sec:WhyBoring}

Writing a specification document can feel ea“We’ll have to do documentation again ? Really !”

If only that could be so easy. Sometimes it is. But usually not. We will go through 4 major difficulties in this process from a simple McDonald’s order to a wedding planning! 

12:15 AM in a McDonald’s fast-food:\\
– Hello, I’d like a Big Mac, a Coke and a large portion of fries please.\\
2 minutes later :\\
– Here is your order, have a nice day.\\
– Thank you, goodbye.

8:30 PM in a fancy restaurant:\\
– Honey, will you marry me ?– Yes !!!\\
One year later, 2 weeks before wedding:\\
– Great Scott! Nothing is going well! The whole table disposition from the wedding planer has to be remade : Anthony cannot bear his brother and my cousins have canceled. All the tablecloths are taupe but they should be beige. We totally forgot to organize the brunch. And you remember your Mexican music band? Wee cannot afford it!!!

\begin{kaobox}[frametitle=Disclaimer]
	This article isn’t in any way a celebration of the well-know fast food license. But it is a very good example of how a customer experience can be fully optimized
\end{kaobox}

Either for a project as quick as a fast food order, or for one as ambitious asa wedding planning, the first step is always the same. We list all that we want and give it to some professionals who will take care of everything.In both cases, we identified 4 characteristics that are very important to master if we want the most seamless communication possible. Let’s dive into them.


\section{The context}

In the McDonald’s order placement scenario, the context issue is solved by your presence into the restaurant. If, by any odd circumstance, you would order a burger in a shoe shop for example you would get a good laugh from the seller or maybe a suspicious look.

In the wedding case, we observe that we haven’t explain enough of the family situation to the wedding planer: the enemies brothers and the not so reliable cousins.

It is very important to give the most complete context for a project to our supplier, to detail our activity. Like that, even before entering the core of your need, he will be able to tell if he is qualified enough for the job or not. To do so, you can present:

\begin{itemize}
	\item Your company, its activities, some technical concepts you need your supplier to understand
	\item Your market and your customers
	\item The project your need is about
\end{itemize}


\section{The vocabulary}
At McDonald’s, the vocabulary to order a menu is clearly defined thanks to the panels above the cashiers. They show all the burgers and side dishes available. Their combinations are rather finite and the customers have all the time to build their order while they wait in line.

For wedding, we can have very precise expectations, like beige tablecloths for example. But if we haven’t correctly defined with the planer what we have in mind when we talk about the beige color, we can end up with taupe!

So clearly stating the vocabulary is essential. There is a probability that you will employ the same wording but with a slightly different meaning. That difference can generate big consequences on the budget or the planning by the end of the project before you understand why.

A simple glossary of the most important elements, placed in the opening of the document, can ensure all the stakeholders that they speak the same language. Do not hesitate to update it during the project in any doubt appears.


\section{The maturity}
This is the most common characteristic in the fast food orders: how many customer arrive at the cashier and are still building up their order. But it’s in the name: fast food! If the order isn’t complete, it won’t take the customer too much time to come back with a new order.

Maybe the Mexican band wasn’t necessary for an already beautiful wedding. Or at least, it wasn’t totally affordable. It would probably be wiser to abandon this “requirement” to secure the event.

More seriously, it is difficult to precisely estimate the maturity of a project at the beginning. Is this item relevant? Useful? Is it a “must have” or a “nice to have”? Do we really need a hammer to kill a fly? Or a rolled newspaper is enough? We can employ the 5 why technique: questioning a need 5 times in a row to find out its root, its prime cause.
We can also openly talk about it with our supplier and give him the opportunity to make suggestions, and for us to trust its expertise.


\section{The exhaustiveness}
We are always asked if we want anything more in a fast food. It is to drive us to buy more? It is to be sure we, the customers, are always pleased? I will leave you to this mystery. But this question has the benefit to ensure an higher maturity, completeness of our order, avoiding us to come back for something we would have forgotten.

Oh, oh, oh! We totally forgot to think about the brunch for the day after! We need a whole new budget for it and talk about it with the caterer! This wedding planner is very lame. He should have warned us.

This issue goes hand to hand with the maturity of our needs. Exhaustiveness asks us “Did you think of it all? Have you not forgot something?” Very hard to tell in advance.
To secure those risks and avoid any “holes in the racket” (we say “trou dans la raquette” when something is missing), we can check what we did in previous projects. We can consult their documentation and use it like a starting checklist. Some parts could be completely reusable and some other may need minor updates. In the same mindset, we can build templates for the projects: pre-filled documents with commonly used chapters and checklists. Some will be abandoned, some will be kept, augmented, optimized

Once again, we can brainstorm with our supplier and summon his expertise. His fresh look upon the project can enlighten some dark areas.
